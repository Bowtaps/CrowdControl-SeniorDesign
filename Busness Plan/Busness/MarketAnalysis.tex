% !TEX root = CCBusinessPlan.tex


\chapter{Market Analysis} 

\section{Market Segmentation}

Market segmentation is a marketing strategy, which involves dividing a broad target market into subsets of consumer, businesses, or countries that have, or are perceived to have, common needs, interests, and priorities, and then designing and implementing strategies to target them. Market segmentation strategies are generally used to identify and further define the target customers, and provide supporting data for marketing plan elements such as positioning to achieve certain marketing plan objectives. (Wiki)

In order to understand a very diverse market, analyzing the demographics of the market is very helpful. The target market for Crowd Control is young adults, between the ages of 21 and 29 years old. The following information will validate the reasoning behind the company’s choice of target market

\section{Demographics}

According to a survey by RJI Mobile in 2014, 53\% of smartphone owners are male and 47\% of smartphone owners are female. The average age of male smartphone users is 41. 24\% of male users are 55+ years old, 35\% of male users are 35-54 years old, and 41\% of male users are 18-34 years old. The average age of female smartphone users is 42. 25\% of female users are 55+ years old, 38\% of female users are 35-54 years old, and 38\% of female users are 18-34 years old. The following chart breaks down smartphone market share by age, operating system and gender. 

According to Google (Appendix XXX), age groups 18-24 and 25-34 tend to notice mobile advertisements more than older age groups, which is good news for our sponsors. The target market fits our app well, due to the likelihood of user interaction with sponsors. In order to stem growth, we need to promote as much interaction between our users and our sponsors as possible.  

Our team will be very selective early on when choosing sponsors for our mobile app. In order to make sure the app promotes growth, our team needs to make sure we understand our customer’s interests and income so we can’t tailor the app to our target market.

According to Pew Research Center in 2015 smartphone ownership is highest among young adults with high income and education levels. In terms of education level: 52\% of HS graduates or less own a smartphone, 69\% of people who took some colleges course own a smartphone, and 78\% of college graduates own a smartphone. It terms of income level: 50\% of people that make \$30,000 per year or less own a smartphone, 71\% of people that make \$30,000-\$49,999 per year own a smartphone, 72\% of people that make \$50,000-\$74,999 per year own a smartphone, and 84\% of people that make \$75,000 or more per year own a smartphone.


\section{Sizing up the Market}

The following chart was built to estimate the overall market size of smartphone devices in the United States for 2015 through 2020. According to the United States Census Bureau (Appendix XXX), the United States is projected to have a population size of 321,369,000 by the end of 2015 and population size of 334,503,000 by the end of 2020. The population will continue to increase at a decreasing rate. Considering the given projected population size at the end of 2015 and the given projected population size at the end of 2020, the yearly population percent changes were assumed to be: .82\%(2016), .81\%(2017), .80\%(2018), .80\%(2019), and .79\%(2020). The previously listed percentages were used to estimate the population size of the United States from 2016 to 2019. According to Statista’s projections (Appendix XXX), 70.1\% of US citizens will own a smartphone in 2015, 75.3\% will own a smartphone in 2016, and 79.7\% will own a smartphone in 2017. Considering the users projections from Statista, the following assumptions were calculated to estimate the user percentage change in the US from 2018 to 2020: 3.5\%(2018), 2.6\%(2019), 1.7\%(2020). The previously listed percentages were used to estimate the total percent of smartphone users from 2018 to 2020. By using the all of the previously listed assumptions the following chart was built. 



%image goes here


Understanding the overall market size of smartphone devices is very important for strategic planning. The following chart will narrow down the data to fit our target market, 21-29 year olds. According to the Nielson demographic chart (Appendix XXX), 85\% of 18-24 year olds own a smartphone and 86.2\% of 25-34 year olds own a cell phone. Due to the fact our target market is for 21-29 year olds, based on the data from Nielson, we will assume, conservatively, 85.5\% of 21-29 year olds own a cell phone. According to the United States Census Bureau (Appendix XXX), the United States is projected to have a population size of 22,740,000 of 20 to 24 year olds by the end of 2015 and a population size of 22,059,000 of 20 to 24 year olds by the end of 2020. Due to the fact the Census Bureau included 20 year olds in the projections, we will assume if we take each project multiplied by 80\% we will have an accurate forecast of 21 to 24 year olds. According to the Census Bureau the United States is projected to have a population size of 22,473,000 of 25 to 29 year olds by the end of 2015 and a population size of 23,722,000 by the end of 2020. We can combine the projections for 21 to 24 years olds and 25 to 29 year olds to form a population projection for 21 to 29 year olds for 2015 and 2020. Considering the projection size of 21 to 29 year olds for 2015 and 2020 the yearly population percent change for the age range was assumed to be: .35\%(2016), .35\%(2017), .34\%(2018), .34\%(2019) and .34\%(2020). The previously listed percentages were used to estimate the population size for 21-29 year olds from 2016 to 2019. The percentages were also used to calculate the estimated percentage of 21 to 29 year old smartphone users, starting with the base amount of 85.5\%, the percent we assumed above, based on the Neilson demographic chart. This is shown below:
 

% image goes here


\section {Assement of competition}


% table goes here


\section{SWOT Analysis}

A SWOT Analysis is a useful technique for understanding and identifying the Strengths, Weaknesses, Opportunities and Threats of a business. \\
SWOT:\\

\textbf{Strengths}
\begin{itemize}
	\item Human Resources
	\item Low Barriers to Entry
	\item Low Startup Costs
	\item Low Fixed Costs
	\item Synergy with SDSM\&t
\end{itemize}
\\
\textbf{Weaknesses}
\begin{itemize}
	\item Rapid Industry Change
	\item User's Cost of Switching
	\item Rivalry Among Existing Competitors
	\item Shifting Threat of new Entry
	\item Shifting Threat of Substitution
	\item Generating Profit FRom New Innovations
	\item Evolving Industry
	\item Financial Resources
	\item Physical Resources
\end{itemize}
\\
\textbf{Opportunities}
\begin{itemize}
	\item Complementry Products and Services
	\item Tehcnological Innovation
	\item Capacity
	\item Forecasted Industry Growth Rate
	\item Many Potential Sponsors
	\item No Substiture Products 
	\item Building Alliances
	\item Technologica Resources
\end{itemize}
\\

\textbf{Threats}
\begin{itemize}
	\item Threat of new Entry
	\item Expected Retaliation from Competitors
	\item New Technology
	\item Radical Industry Change: fom Threat of Obsolescene
\end{itemize}