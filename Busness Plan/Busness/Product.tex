% !TEX root = CCBusinessPlan.tex


%\setlist{nolistsep}
\definecolor{green}{HTML}{66FF66}
\definecolor{myGreen}{HTML}{009900}

\renewcommand{\familydefault}{\sfdefault}
\renewcommand{\arraystretch}{1.5}

\chapter{The Product}

\section{Introduction}

Crowd Control is a mobile application designed to “ease the experience of going out.” Crowd Control seeks to provide users involved in nightlife events, concerts, festivals, and any other group activities, mobile technology to add value to their overall experience. 

\section{Third Party Requirements}

This application is designed to be a background tool. It has to handle importing and exporting messaging and GPS data to and from the user. It has to keep this information up-to-date as well as quickly accessible. It also has to store this information in a safe and secure way.
For the side being developed in Android. We have handled the following problems though the use of services, and the UX design of the app. For storing the information, we are using a back-end service called Parse. Parse encrypts all of its data and gives great functionality to a database. Another service being used for the messaging, Sinch. Sinch also uses encryption to protect app-to-app messages.
The goal of our UX is to make it so that the user the doesn't have to think about the about. We accomplish this by implementing a one-time log in for the user. Another way we will be streamlining things is by keeping the app at the group screen when launched when the user is in said group.\\

\noindent
\textbf{Parse}

Parse is a back-end database service that is the backbone of this app. The entirety of its database is encrypted for our users protection. Inside Parse we are using several tables. These tables abstract user information so that even the displayed information is limited to protect the user.\\

\noindent
\textbf{Sinch}

Sinch is a back-end messaging system. It provides encrypted device-to-device messaging. Though sinch, we will be able to maintain a more stable messaging platform.




\section{Developement Requirements}

An important question we need to ask ourselves right away is what type of application we want to develop. The two basics types to choose from are a native version and a cross platform development.  The following information pulled directly from uxmag.com will outline the differences, strengths, and drawbacks of both types.
A native app is one that is built for a specific platform, such as iPhone or Android, using their code libraries and accessing their available hardware features (camera, GPS, etc). A cross platform compiler, such as Xamarin, allows for simultaneous development on both platforms but does have its drawbacks.. Let's explore the pros and cons of both approaches.



\begin{center}
\begin{tabularx}{\textwidth}[t]{m{4cm} X}
%table spot one
\arrayrulecolor{green}\hline
\textbf{\textcolor{myGreen}{Native App Strengths}} & \\
\hline
Speed.  &
\begin{minipage}[t]{\linewidth}%
\begin{itemize}
\item[1.1] 
Native apps tend to be faster and more responsive. Because the code that runs the app is stored locally on the phone, there is no time spent waiting for static content (such as images and text) to be downloaded from the web. While dynamic content may still need to be accessed from the web, it’s an improvement over the web-based model in which everything needs to be downloaded each time.\\

\end{itemize} 
\end{minipage}\\

\arrayrulecolor{black}\hline

Local Storage &
\begin{minipage}[t]{\linewidth}%
\begin{itemize}
\item[1.2] Native apps can run asynchronously, meaning dynamic information can be stored locally on the phone temporarily and synced with the central web-based server later. While new technologies and features (such as those in Xamarin) will allow for this to also be done in cross platform environment it is not as reliable as native.\\
\end{itemize} 
\end{minipage}\\

\hline

Killer Features &
\begin{minipage}[t]{\linewidth}%
\begin{itemize}
\item[1.3]  Going with the native app approach gives you access to that platform's hardware features allowing interesting functionalities such as taking photos, accessing GPS information, making phone calls, leveraging near field communication (NFC), etc. Because web-based apps are platform agnostic, they do not have access to the device’s hardware features.\\
\end{itemize}
\end{minipage}\\

% section 2
\arrayrulecolor{green}\hline
\textbf{\textcolor{myGreen}{Native App Draw Backs}} \\
\hline

Coding &
\begin{minipage}[t]{\linewidth}%
\begin{itemize}
\item[2.1] The biggest drawback to developing a native app vs. a cross platform one is that a separate code base must be created and maintained for each individual platform. For example, if you decided to initially build an iPhone app, you would have to design, code, and deploy an iOS app to the App Store. If you then decide down the road that you also want an Android version, you will have to redesign the app for the Android device, code and deploy it to the Android app store—likewise for other platforms. From a development perspective, the code bases are two entirely different languages and will have to be completely rewritten simply to mimic the original app’s functionality.\\
\end{itemize}
\end{minipage}\\

% section 3
\hline
\multicolumn{2}{l}{%
\textbf{\textcolor{myGreen}{Cross Platform Strengths}}} \\
\hline

Single Solution &
\begin{minipage}[t]{\linewidth}%
\begin{itemize}
\item[3.1] The biggest upside to a cross platform development approach is, of course, the biggest downside to a native one. When developing a cross platform you are centralizing your offering. Single-source means that there is a single version of the code base that all users across all platforms access and use.\\
\end{itemize} 
\end{minipage}\\

Single Code Updates &
\begin{minipage}[t]{\linewidth}%
\begin{itemize}
\item[3.2] With the language being written in a single language allows for faster updates and new features to be added.  \\
\end{itemize} 
\end{minipage}\\



%section 4
\hline
\multicolumn{2}{l}{%
\textbf{\textcolor{myGreen}{Cross Platform Draw Backs}}} \\
\hline

Hardware &
\begin{minipage}[t]{\linewidth}%
\begin{itemize}
\item[3.1] With cross platform development does not always properly implement the features of the native language. For the things that require specific native code you have to create translators to go between the cross code and the native code.
\end{itemize} 
\end{minipage}\\


\end{tabularx}
\end{center}

\noindent
\textbf{Our Decision}

Both approaches certainly have their share of benefits as well as drawbacks. A long-term strategy would seem to favor web-based over native apps, but no matter which approach is taken, a well-orchestrated user experience is the best defense in the rapidly evolving world of mobile platforms and devices. We have decided to use the native approach because it allows for a more 

\section{Product Description}

Crowd Control is a mobile application, which aims to add value to the overall experience of event goers though group management, integrated group messaging, and gps locations. All of these features will be bundled into a easy to use mobile application that allows for everything your group needs to know to be in one location at all times.

\subsection{Overview}

The application was built to serve three primary aspects of crowd control: 
\begin{itemize}
\item Event-based group management
\item Integrated group chat
\item Opt-in periodic location updates
         (Detailed)
\end{itemize}

\subsection{Features}

\textit{\textbf{Group Management:}}\\

The application will allow users to create temporary groups with known and unknown users. The groups will disband after an event is over, allowing a more dynamic experience. \\

\noindent
\textit{\textbf{Group Messaging:}} \\

The mobile application will feature an integrated messenger, which removes the need for users to resort to third-party services to communicate with group members. Along with third party messaging apps ( ones that are outside of the app ) it allows for ease because it eliminates the issues associated with group messaging such as, cross platform messaging, cross carrier messaging, and time stamping issues. \\

\noindent
\textit{\textbf{GPS Tracking:}}\\

Many third-party applications use the GPS feature to help track other users. Our GPS tracking is designed with groups and battery life in mind. Our implementation would be less demanding on the users’ batteries by only sending location updates at customizable time intervals or when requested. Because the groups are temporary, tracking stops after the group is disbanded. \\






