% !TEX root = CCBusinessPlan.tex


\setlist{nolistsep}
\definecolor{green}{HTML}{66FF66}
\definecolor{myGreen}{HTML}{009900}

\renewcommand{\familydefault}{\sfdefault}
\renewcommand{\arraystretch}{1.5}

\chapter{The Product}

\section{Introduction}

Crowd Control is a mobile application designed to “ease the experience of going out.” Crowd Control seeks to provide users involved in nightlife events, concerts, festivals, and any other group activities, mobile technology to add value to their overall experience. 

\section{Developement Requirements}

An important question we need to ask ourselves right away is what type of application we want to develop. The two basics types to choose from are a native version and a cross platform development.  The following information pulled directly from uxmag.com will outline the differences, strengths, and drawbacks of both types.
A native app is one that is built for a specific platform, such as iPhone or Android, using their code libraries and accessing their available hardware features (camera, GPS, etc). A cross platform compiler, such as Xamarin, allows for simultaneous development on both platforms but does have its drawbacks.. Let's explore the pros and cons of both approaches.



\begin{center}
\begin{tabularx}{\textwidth}[t]{XX}
\arrayrulecolor{green}\hline
\textbf{\textcolor{myGreen}{Goal 1 Eradicate Extreme Poverty}} & \\
\hline
Target 1.A Halve, between 1990 and 2015, the proportion of the people whose income is less than \$1 a day. & 
\begin{minipage}[t]{\linewidth}%
\begin{itemize}
\item[1.1] Proportion of population below \$1 purchasing power parity (PPP) a day$^a$
\item[1.2] Poverty Gap ratio [incidence x depth of poverty]
\item[1.3] Share of the poorest quintile in national consumption
\end{itemize} 
\end{minipage}\\

\arrayrulecolor{black}\hline

Target 1.B Achieve full and productive employment and decent work for all, including women and young people &
\begin{minipage}[t]{\linewidth}%
\begin{itemize}
\item[1.4] Growth of GDP per person employed 
\item[1.5] Employment to population ratio
\item[1.6] Proportion of employed people living below \$1 (PP) a day
\item[1.7] Proportion of own-account and contribution family workers in total employment
\end{itemize} 
\end{minipage}\\

\hline

Target 1.C Halve, between 1990 and 2015, the proportion of people who suffer from hunger &
\begin{minipage}[t]{\linewidth}%
\begin{itemize}
\item[1.8] Prevalence of underweight children under five years of age
\item[1.9] Proportion of population below minimum level of dietary energy consumption
\end{itemize}
\end{minipage}\\

\arrayrulecolor{green}\hline
\textbf{\textcolor{myGreen}{Goal 2 Achieve universal primary education}} \\
\hline

Target 2.A Ensure that by 2015 children everywhere, boy and girls alike, will be able to complete a full course of primary schooling. &
\begin{minipage}[t]{\linewidth}%
\begin{itemize}
\item[2.1] Net enrollment ratio in primary education
\item[2.2] Proportion of pupils starting grade 1 who reach last grade of primary education
\item[2.3] Literacy rate of 15- to 24-year-olds, women and men
\end{itemize}
\end{minipage}\\

\hline
\multicolumn{2}{l}{%
\textbf{\textcolor{myGreen}{Goal 3 Promote gender equality and empower women}}} \\
\hline

Target 3.A Eliminate gender disparity in primary and secondary education, preferably by 2005, and in all levels of education no later than 2015 &
\begin{minipage}[t]{\linewidth}%
\begin{itemize}
\item[3.1] Ratios of girls to boys in primary, secondary and tertiary education
\item[3.2] Share of women in wage employment in the non-agricultural sector.
\end{itemize} 
\end{minipage}
\end{tabularx}
\end{center}


\textbf{Our Decision} \\

Both approaches certainly have their share of benefits as well as drawbacks. A long-term strategy would seem to favor web-based over native apps, but no matter which approach is taken, a well-orchestrated user experience is the best defense in the rapidly evolving world of mobile platforms and devices. We have decided to use the native approach because it allows for a more 

\section{Product Description}

\subsection{Overview}

\subsection{Features}








