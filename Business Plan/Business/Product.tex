% !TEX root = CCBusinessPlan.tex


%\setlist{nolistsep}
\definecolor{green}{HTML}{66FF66}
\definecolor{myGreen}{HTML}{009900}

\renewcommand{\familydefault}{\sfdefault}
\renewcommand{\arraystretch}{1.5}

\chapter{The Product}

\section{Introduction}

Crowd Control is a mobile application designed to "ease the experience of going out" by providing
simple yet powerful group communication tools. Crowd Control seeks to help users with easy-to-use
group messaging and organizational features. The product is most useful in loud and crowded places
where direct communication with group members is difficult, such as concerts, festivals, and
amusement parks.



\section{Third Party Requirements}

Crowd Control is designed to be a background tool. It has to handle instant messaging, retrieving and
securely distributing GPS data to other group members, and doing so while minimizing power
consumption and network access. The information must be kept up-to-date and be quickly accessible.
Additionally, all of this personal user information must be stored and communicated securely. Using
capable third party services for storing and transmitting data and well-designed user interfaces, Crowd
Control is capable of meeting all of these requirements.

For storing data, we are using Parse as our back-end service. Data stored on Parse servers is
encrypted while simultaneously being easily accessible using the provided developer APIs. For instant
messaging, Crowd Control will be using the Sinch messaging service. Sinch also uses encryption to
protect app-to-app messages.

The goals of our user experience designs are to keep the app simple to learn, easy to use, and fun to
experience. We accomplish these goals by employing modern UI design techniques, keeping the
application as responsive and straightforward as possible, and by streamlining and simplifying
important tasks. For example, Crowd Control only requires the user to sign up or log in one time,
remembering who the user is from that point on. Another example is automatically displaying only the
most relevant information; when a user is currently part of a group, the context of the application will
change to be entirely focused on the group.\\

\noindent
\textbf{Parse}

Parse is a back-end database service that is the backbone of Crowd Control. Parse provides data
storage and retrieval services, cloud functionality, and customizability that will allow our application to
function exactly as intended. The entirety of the data stored in Parse is encrypted for our users'
protection. Additionally, our own database designs allow user messages and geolocations to be
encrypted end-to-end, meaning that no one other than group members---not even ourselves---can view
messages and location data shared between our users.\\

\noindent
\textbf{Sinch}

Sinch is a back-end messaging service that allows us to provide group messaging to our users without
needing to reimplement such a system ourselves. Instant messages sent through Sinch are encrypted
end-to-end, meaning that even if the data was intercepted, it would be unreadable except by those
whom the message was intended for. Fast message passing, push notifications, and secure
communication are the features promised by this service, and Crowd Control will be making full use of
Sinch to give our users the best experience possible.



\section{Developement Requirements}

One of the most critical decisions to make when building a mobile application is to choose the target
platform. Applications built for as many platforms as possible can reach a wider market, but such
projects can be expensive and technically challenging to successfully undertake. There exist tools that
allow applications to be developed for multiple platforms simultaneously, but they bring with them a
different set of difficulties and limitations. The chart on the next page displays information pulled directly from \url{uxmag.com} that outlines the key differences, strengths, and drawbacks of multi-platform
development using native technologies versus cross-platform tools.\\

\noindent
\textbf{Our Decision}

There is no question that Crowd Control must exist on multiple platforms; the combined popularity of
the top smartphone platforms creates a target market that we cannot afford to ignore. This boils our
decision down to choosing between developing Crowd Control using native technologies and
building it using cross-platform tools and services. Both approaches certainly have their share of
benefits as well as drawbacks. A long-term strategy would seem to favor web-based over native apps,
but no matter which approach is taken, a well-orchestrated user experience is the best defense in the
rapidly evolving world of mobile platforms and devices. Additionally, cross-platform tools would carry
an additional development cost and applications built with these tools are subject to feature limitations
that native applications are not bound by.

Weighing the pros and cons between developing a native application or a cross-platform application,
we have determined that Crowd Control should be developed natively. Despite the additional time and
potential effort that this would require, it will enable us to build an overall better experience for our users
and will avoid a potentially risky dependency on another project that lies outside of our control.



\begin{center}
\begin{tabularx}{\textwidth}[t]{m{4cm} X}
%table spot one
\arrayrulecolor{green}\hline
\textbf{\textcolor{myGreen}{Native App Strengths}} & \\
\hline
Speed.  &
\begin{minipage}[t]{\linewidth}%
\begin{itemize}
\item[1.1] 
Native apps tend to be faster and more responsive. Because the code that runs the app is stored
locally on the phone, there is no time spent waiting for static content (such as images and text) to be
downloaded from the web. While dynamic content may still need to be accessed from the web, it's an
improvement over the web-based model in which everything needs to be downloaded each time.\\

\end{itemize} 
\end{minipage}\\

\arrayrulecolor{black}\hline

Local Storage &
\begin{minipage}[t]{\linewidth}%
\begin{itemize}
\item[1.2] Native apps can run asynchronously, meaning dynamic information can be stored locally on
the phone temporarily and synced with the central web-based server later. While new technologies and
features (such as those in Xamarin) will allow for this to also be done in cross platform environment it is
not as reliable as native.\\
\end{itemize} 
\end{minipage}\\

\hline

Killer Features &
\begin{minipage}[t]{\linewidth}%
\begin{itemize}
\item[1.3]  Going with the native app approach gives you access to that platform's hardware features
allowing interesting functionalities such as taking photos, accessing GPS information, making phone
calls, leveraging near field communication (NFC), etc. Because web-based apps are platform agnostic,
they do not have access to the device's hardware features.\\
\end{itemize}
\end{minipage}\\

\end{tabularx}
\end{center}

\begin{center}
\begin{tabularx}{\textwidth}[t]{m{4cm} X}

% section 2
\arrayrulecolor{green}\hline
\textbf{\textcolor{myGreen}{Native App Draw Backs}} \\
\hline

Coding &
\begin{minipage}[t]{\linewidth}%
\begin{itemize}
\item[2.1] The biggest drawback to developing a native app vs. a cross platform one is that a separate
code base must be created and maintained for each individual platform. For example, if you decided to
initially build an iPhone app, you would have to design, code, and deploy an iOS app to the App Store.
If you then decide down the road that you also want an Android version, you will have to redesign the
app for the Android device, code and deploy it to the Android app store---likewise for other platforms.
From a development perspective, the code bases are two entirely different languages and will have to
be completely rewritten simply to mimic the original app's functionality.\\
\end{itemize}
\end{minipage}\\

\end{tabularx}
\end{center}

\begin{center}
\begin{tabularx}{\textwidth}[t]{m{4cm} X}

% section 3
\arrayrulecolor{green}\hline
\multicolumn{2}{l}{%
\textbf{\textcolor{myGreen}{Cross Platform Strengths}}} \\
\hline

Single Solution &
\begin{minipage}[t]{\linewidth}%
\begin{itemize}
\item[3.1] The biggest upside to a cross platform development approach is, of course, the biggest
downside to a native one. When developing a cross platform you are centralizing your offering.
Single-source means that there is a single version of the code base that all users across all platforms
access and use.\\
\end{itemize} 
\end{minipage}\\

Single Code Updates &
\begin{minipage}[t]{\linewidth}%
\begin{itemize}
\item[3.2] With the language being written in a single language allows for faster updates and new
features to be added.  \\
\end{itemize} 
\end{minipage}\\

\end{tabularx}
\end{center}

\begin{center}
\begin{tabularx}{\textwidth}[t]{m{4cm} X}
%section 4
\arrayrulecolor{green}\hline\hline
\multicolumn{2}{l}{%
\textbf{\textcolor{myGreen}{Cross Platform Draw Backs}}} \\
\hline

Hardware &
\begin{minipage}[t]{\linewidth}%
\begin{itemize}
\item[3.1] With cross platform development does not always properly implement the features of the
native language. For the things that require specific native code you have to create translators to go
between the cross code and the native code.
\end{itemize} 
\end{minipage}\\

\end{tabularx}
\end{center}



\section{Product Description}

\subsection{Overview}

Crowd Control is a mobile application that aims to ease the experience of going out by providing group
management, group messaging, and GPS tracking features. All of these useful tools are bundled into a
single, easy-to-use mobile application that is easy to set up, simple to learn, and enjoyable to use. With
Crowd Control, users will be able to keep in communication and find each other, even when in loud and
crowded places.

\subsection{Features}

The application is built to provide the following core features:

\begin{itemize}
\item Event-based group management
\item Integrated group instant messaging
\item Optional periodic location updates
\end{itemize}

These features, while essential to the product, are far from the only features that are in development.
Crowd Control is built to grow with its customers, and future versions will contain additional features
that will compliment the rest.\\

\noindent
\textit{\textbf{Group Management}}

Crowd Control will allow users to create temporary "groups" that they can invite their friends and family
to join. Once joined, members of this group can take advantage of the other features of the app for as
long as the group is in use. After an amount of time specified by the group leader or if the group
becomes inactive and unused for several hours, then the group will be "disbanded", removing all
users from the group and unlinking their profiles from one another. Leaders can manually disband a
group whenever they desire.

The flexible nature of group management will allow users to join groups without concern for who
might also be in the group, as personal data will never be automatically shared with the other users
in the group without consent. This offers several advantages over, say, exchanging phone numbers.\\

\noindent
\textit{\textbf{Instant Messaging}}

Crowd Control will feature integrated instant messaging, which removes the need for users to resort to
alternate communication platforms such as SMS to communicate with group members. This also
contributes towards the overall goal of ease of use because it eliminates the need for members to
use multiple apps to stay in contact; all the needed features in a single bundle. Additionally, because
the messaging is group-based and built right into the application, there's no need for group members
to exchange permanent contact information with other members they may not know.\\

\noindent
\textit{\textbf{GPS Locating}}

In many social situations, noise levels and other factors can distract people so that they cannot hear
or respond to phone calls or messages from those they are with. Crowd Control solves this problem by
automatically sharing each user's geographical location with the other members of the group. This way,
if the group cannot find or contact another group member, they could use this application to locate
them.

Because personal privacy is one of the top priorities of Crowd Control, individual location gathering is
performed on each user's own device and encrypted so that no one but the other members in the group
can read the information. Additionally, location tracking features are entirely optional and can be
disabled on a per-user and per-group basis.

Many third-party applications use the GPS feature to help track other users. Our GPS tracking is
designed with battery life in mind; it is important that Crowd Control can operate without significant
power use, as the target use cases involve users straying far from any power source. Our implementation
is less demanding on the user's battery by only sending location updates at customizable time intervals
or when requested. Because the groups are temporary, tracking stops after the group is disbanded or if
the user force-closes the application.
