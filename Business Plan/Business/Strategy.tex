% !TEX root = CCBusinessPlan.tex


\chapter{Strategy and Implementation Summary}

The initial release of Crowd Control will be in conjunction with local businesses in the Rapid City and Black Hills area. This will allow those businesses to gain promotion from our app, and widen our user base.

\section{Market Strategy}
As there are many alternatives to our product, our primary mission is to show that Crowd Control can be the all-in-one package that users will need. Since no apps currently encapsulate all the things that our product will, it is important for us to play to our strengths.

We will be working closely with local businesses to help promote events and use those events as opportunities to showcase what our app does and why Crowd Control is their next necessary tool when going to these events.

The Rapid City/Black Hills area is perfect for an initial release, as it has a large tourist influx and hosts many local events in and around the hills. The spaciousness of the area is another opportunity to showcase our GPS location features, and help validate that, if this app is in a larger city, it will scale well.

It is critical to keep the relationship with businesses strong, as this is one of the features that will set us apart from our competitors. Many different services that do focus on event management don’t promote businesses.

\section{Sales Strategy}

Crowd Control’s largest barrier to entry (as is with most mobile applications), is the price; our app must remain free to retain the largest amount of users, as most people will not download an app if they have to pay for it beforehand.

Since Crowd Control will be free, revenue will come from the connection we have with businesses and different event coordinators. The business-sponsored suggestions we serve to our users will be the primary source of revenue, as this will connect our users with our sponsors.


\subsection{In-App Pricing}
Although Crowd Control is a free-to-download application, users can purchase access to customized styles and other cosmetic enhancements, such as layouts, emojis, and icons. \\

Additionally, since the the maximum number of people allowed in a group is 8 people, users can choose to make a one-time purchase to double the capacity to 16 people for the lifetime of their account. To further expand group sizes, users can purchase instant group capacity upgrades to expand beyond their current limit. For example, if a user wanted to host a group of 25 people but their current limit is 8 people, they could purchase the instant group capacity upgrade. This would extend their capacity to 25 people for the duration of the group. After the group disbands, that capacity will be reduced back to their original capacity. The group size pricing will increment by 5 people at a rate of \$0.20 per person.

%TODO: Fix line separation in URL
\subsection{Advertisement Pricing}
Our advertisement model is based on a CPM (cost-per-thousand impressions) model. According to \url{marketingterms.com}, the total price paid in a CPM deal is calculated by multiplying the CPM rate by the number of CPM units. For example, one million impressions at \$10 CPM equals a \$10,000 total price. This is illustrated in the example below:

\begin{center}
$1,000,000 / 1,000 \textrm{ impressions} = 1,000 \textrm{ units}$\\
$1,000 \textrm{ units } \times \$10 \textrm{ CPM} = \$10,000 \textrm{ total price}$
\end{center}

Using the CPM model, our advertisement prices will scale from the number of users in a given area. We don't currently have a set CPM rate, but plan to start around \$3.00.

