% !TEX root = DesignDocument.tex
\definecolor{green}{HTML}{66FF66}
\definecolor{myGreen}{HTML}{009900}
\renewcommand{\arraystretch}{1.5}

\chapter{System and Unit Testing}

Crowd Control utilizes multiple sevices across different platforms, requiring
all parts to be operating correctly in order to provide consistent service to
all users. We use system and unit testing to assist developers in identifing 
potential issues and to verify that the product being produced meets
requirements. The goal of system and unit testing is to automate the process to
increase development efficiency and reduce testing errors.

Given the time constraints and scope of Crowd Control, we have been forced to
make compromises in regards to system and unit testing. Implementing and running
automated unit testing, can be as time-consuming as writing production code. In
fact, it is extremely common for a single function of production code to require
many times as many lines of unit testing code to completely test the function.
While automated testing provides many invaluable benefits, it would ultimately be
a hinderance to the project if it means missing critical deadlines and not
achieving crucial project milestones.

In this section, we describe our approach to testing, our goals, and the test
cases that we have developed and implemented thus far. We also discuss the
future plans for testing and how we will be able to devote more time and energy
towards this project in the upcoming months.


\section{Overview}

For this project, we use a combination of ``black-box'' and ``white-box''
testing. This means that there are some aspects of the project that we have
created ourselves and have full control over its execution, and there are others
that we have little control over because it is was provided by a third party.
In practical terms, it means that we can test our own code with the full
knowledge of how it is intended to work, allowing us to design tests that target
potential problem areas. When designing tests that make use of a third-party
service or a library that we have not written ourselves, we can only write tests
that verify that the supplied functionality works as documented.

Because one of the goals of this project is to develop the applications using
native languages and technology, we make use of the most established testing
frameworks for the platform we are testing on. For the iOS application, we use
the testing tools and frameworks built directly into Xcode. For the Android
application, we use Espresso for testing our interface and JUnit for testing our
classes and methods. To test Parse, the third party service we use for data
storage and retrieval, we use unit testing on both Android and iOS, as
interacting with Parse requires a running application on a device. To test
Sinch, the third party service we use for instant messaging between users, we
use the unit testing frameworks we use for testing our Android and iOS
applications since interacting with Sinch also requires a running instance of
the application on a device.

Running unit tests in all of the frameworks above yields a list of binary
results; for each test executed, we receive either a ``pass'' or ``fail''
result. If a test fails, we may also receive a message describing the issue
that caused the test to fail. There exist several conditions that could cause
a test to fail.

\begin{itemize}
	\item An uncaught exception is thrown
	\item A runtime error is encountered
	\item The application crashes
	\item A test assertion fails
\end{itemize}

When designing test cases, we begin with the set of user stories identified at
the beginning of the project. A list of user stories can be found in the section
\ref{userStories}. From these user stories, we build a testing matrix that will
be referenced when writing our automated tests. Tests are then written as the
code they are intended to test is written.

Tests are also to be run relatively frequently. Ideally, developers would enable
the set of tests relevant for the aspect of the project they are currently
working on and run those tests as needed during development and again before the
code is merged in with the rest of the project. Then, the full set of tests
would be run automatically and at a time that would not interfere with
development or execution of the product.


\section{Dependencies}

This project can be broken into several pieces that all work together to form a
single cohesive product. Taken individually, these pieces would be of little
utility. It is important that we test each how well each component interacts
with each other as well as each isolated part as best we can. The four main
components of this project are as follows:

\begin{itemize}
	\item Android client application
	\item iOS client application
	\item Parse service and cloud code
	\item Sinch service
\end{itemize}

Each of these aspects of the project need to be somehow tested, although some
parts can be more accurately tested than others. More thorough requirement
breakdowns of each aspect are described in the subsections below.


\subsection{Android Client Application}

To test our client code running locally on Android devices, we use two testing
frameworks: JUnit for testing Android-independent code and Espresso for testing
navigation and user interface requirements. These frameworks are run from within
our integrated development environment, Android Studio. Additionally, for
interface testing, we require a running emulator or a connected Android device on
which Android Studio can run the unit tests.

While developing unit tests for the JUnit and Espresso frameworks, we use
\href{https://google.github.io/android-testing-support-library/docs/index.html}
{Google's Android testing Support Library Documentation} website. This resource
provides accurate information on installing and using these frameworks in
conjunction with Android Studio. Writing unit tests and best practices, as well
as class documentation are available from this website.


\subsection{iOS Client Application}

To test our client code running locally on iOS devices, we use the testing tools
included in Xcode, our iOS integrated development environment. Backend unit
testing and interface testing are all handled by the same framework. These tests
are run from within Xcode. Interface tests require a running iOS emulator or an
iOS device connected to the computer. Because the unit testing tools used to
test iOS devices are tied to Xcode, this means that the tests can only be run on
a Macintosh computer.

While developing unit tests for iOS devices, we reference \href{https://developer.apple.com/library/ios/documentation/ToolsLanguages/Conceptual/Xcode_Overview/UnitTesting.html}
{iOS Developer Library} website. This resource contains instructions on
designing and setting up unit tests, as well as instructions on running unit
tests. Class documentation can also be found on this website.


\subsection{Parse Service and Cloud Code}

Some aspects of our project exist as ``cloud code'', which is custom server-side
logic executed by our database backend service, Parse. This cloud code is
written in JavaScript and handles database operations that require security,
data integrity, consistency, and efficency. However, because this cloud code
runs on a remote server to which we have limited access and concealed knowledge
of its internal workings, we are forced to test the our code by the only means
we have; using the client applications.

Parse does not supply testing features, and thus we must devise a way to execute
our own unit tests using the tools we have. To test the functionality of our
cloud code and our Parse interaction code, we require these things:

\begin{enumerate}
	\item The Parse libraries installed
	\item Parse interactivity built into the application
	\item An active Internet connection to the Parse service
	\item An emulator/simulator/developer device
	\item The testing framework associated with the client platform being used
	for testing
	\item A Parse application key
	\item A separate Parse database dedicated to testing
\end{enumerate}

In order to run tests on our cloud code, we use both versions of our client (iOS
and Android) to connect to the Parse server, perform database operations by
calling cloud code, then verifying the results. We use the aforementioned
testing frameworks to develop and execute these tests on both platforms. These
tests can be performed on device simulators, device emulators, and physical
developer devices alike.

When using client software to test server software, we must properly configure
the clients to interact with the database as if it were no different than a
production database. Thus, we require that the Parse libraries be included in
the project and that actual Parse functionality built into the application. As
with all Parse-enabled applications, this requires a unique application key be
assigned by Parse and used to access the database.

Because running tests on Parse result in changes to the data in the database we
are testing with, it is crucial that the automated tests be run on an
independent Parse database dedicated to testing. Running such tests on either
the production or development databases could result in irreversable destruction
of data that we cannot afford to jeapordize. In order to avoid this, all unit
tests that involve testing Parse functionality should be executed using a
database where the data is used exclusively for testing.


\subsection{Sinch Service}

Another crucial third party service we utilize is Sinch, an instant messaging
platform that we use in conjunction with Parse. This service does not have cloud
code support, but that is okay since we have no need for cloud code here.
However, it is important that we test how well our client applications connect
to and use the service to ensure that communications run smoothly.

In order to test the functionality of Sinch and how well our applications
interact with it, we require these things:

\begin{enumerate}
	\item The Sinch libraries installed
	\item Sinch functionality built into the application
	\item An active internet connection to Sinch
	\item An emulator/simulator/developer device running the application
	\item The testing framework associated with the client platform being used
	for testing
	\item A Sinch application key
	\item A separate Sinch application dedicated to testing
\end{enumerate}

As with Parse, Sinch is a third party service to which we have limited access
and no knowledge of the implementation details. With the addition of the fact
that we have no cloud code for Sinch to test, the only aspect of this service we
need to test is how our client applications interact with the service.

To test our integration with Sinch, we run tests using our Android and iOS
client applications on either device emulators, device simulators, or physical
development devices. To develop and run these tests, we use the aforementioned
testing frameworks used for testing the client code itself on each platform.
We then run our tests on these devices, which cause the application to connect
to and communicate with Sinch using the provided developer API. Sinch requires
an active Internet connection, so the success of the tests depends on the
testing devices having a stable Internet connection.

Lastly, like Parse, Sinch supplies a unique application key which our clients
use to authenticate with the Sinch servers. All versions of our client
applications use this key and is what allows them to communicate with each
other. Also like Parse, we require that all tests run through Sinch also be run
using a special Sinch application key that is completely separate from the
production and development versions of the application. We must do this in order
to avoid user ID conflicts and to limit any operations that may affect service
uptime and reliability.


\section{Test Setup and Execution}

As with most aspects of software development, test setup and execution requires
two distinct phases: design and execution. As functionality and features are
designed and built, correlating tests are designed concurrently and added to our
testing matrix. After a test has been adequately designed with inputs and
expected outputs defined, the unit tests can be implemented in whatever testing
framework associated with the platform which the tests are targetting.

Setting up unit tests involves referencing the previously designed test
parameters while writing code that will run on the target platform within the
selected testing framework. For example, implementing a unit test that will run
on the Android platform, we create a new unit test class for the portion of the
code we aim to test, following framework usage. The specifics on implementing a
unit test within the constraints of a framework often differ greatly between
frameworks.

Essentially, we create a new testing class that extends some framework-supplied
class. We then create functions that will perform the tests under different
conditions, called test cases. Usually, if our goal is to test the functionality
of a single class, we will create a single test class with the sole purpose of
testing said class.

A full breakdown of our most current testing matrix can be found in the
section \ref{systemTesting}.


\section{System Testing} \label{systemTesting}

Most of the unit tests our team has designed are related to the functionality of
the system as a whole, focusing on user experience and how we expect the
application to respond to inputs at each given state. Rather than focus on
testing a specific aspect of the project, this testing matrix includes test
cases and expected results that often require the cooperation of every part of
the project.


\begin{center}
\begin{tabularx}{\textwidth}[t]{p{4cm} X}
\textbf{Test case} & \textbf{Result}\\
\arrayrulecolor{green}\hline
\multicolumn{2}{l}{\textbf{\textcolor{myGreen}{Log in with email}}}\\
\hline
Valid credentials & Success \\
Invalid email & Failure \\
Invalid password & Failure \\
\end{tabularx}
\end{center}

\begin{center}
\begin{tabularx}{\textwidth}[t]{p{4cm} X}
\arrayrulecolor{green}\hline
\multicolumn{2}{l}{\textbf{\textcolor{myGreen}{Log in with Facebook}}}\\
\hline
Valid credentials & Success \\
Invalid credentials & Failure \\
\end{tabularx}
\end{center}

\begin{center}
\begin{tabularx}{\textwidth}[t]{p{4cm} X}
\arrayrulecolor{green}\hline
\multicolumn{2}{l}{\textbf{\textcolor{myGreen}{Log in with Facebook}}}\\
\hline
Valid credentials & Success \\
Invalid credentials & Failure \\
\end{tabularx}
\end{center}

\begin{center}
\begin{tabularx}{\textwidth}[t]{p{4cm} X}
\arrayrulecolor{green}\hline
\multicolumn{2}{l}{\textbf{\textcolor{myGreen}{Sign up with email}}}\\
\hline
Email already in use & Message \\
Valid credentials & Success \\
Invalid password & Failure \\
Unconfirmed password & Failure \\
Invalid email & Failure + Message \\
Whitespace-only username & Failure + Message \\
Empty username & Failure + Message \\
Empty email & Failure + Message \\
Empty password & Failure + Message \\
\end{tabularx}
\end{center}

\begin{center}
\begin{tabularx}{\textwidth}[t]{p{4cm} X}
\arrayrulecolor{green}\hline
\multicolumn{2}{l}{\textbf{\textcolor{myGreen}{Create a group}}}\\
\hline
No name & Message \\
Whitespace-only name & Failure + Message \\
No description & Success + Message \\
Whitespace-only description & Success + Message \\
Else & Success + trim whitespace + collapse whitepace in name \\
\end{tabularx}
\end{center}

\begin{center}
\begin{tabularx}{\textwidth}[t]{p{4cm} X}
\arrayrulecolor{green}\hline
\multicolumn{2}{l}{\textbf{\textcolor{myGreen}{Search for users}}}\\
\hline
Enter display name & Display list of matching users or ``no matches'' \\
Enter email & Detect email, no change \\
Enter phone number & Detect phone number, no change \\
Enter incomplete phone number or incomplete email & Display ``no matches'' \\
\end{tabularx}
\end{center}

\begin{center}
\begin{tabularx}{\textwidth}[t]{p{4cm} X}
\arrayrulecolor{green}\hline
\multicolumn{2}{l}{\textbf{\textcolor{myGreen}{Send invite to users to join group}}}\\
\hline
By email unassociated with user & Generate and send email \\
By email associated with user & Send app notification to user, display invite in-app \\
By phone number unassociated with user & Detect phone number, no change \\
By phone number associated with user & Send app notification to user, display invite in-app \\
\end{tabularx}
\end{center}

\begin{center}
\begin{tabularx}{\textwidth}[t]{p{4cm} X}
\arrayrulecolor{green}\hline
\multicolumn{2}{l}{\textbf{\textcolor{myGreen}{Join a public group}}}\\
\hline
Tap ``join group'' & Displays confirmation message (y/n) \\
Accept confirmation message & Message ``request sent'' \\
Decline confirmation message & Dismiss message \\
\end{tabularx}
\end{center}

\begin{center}
\begin{tabularx}{\textwidth}[t]{p{4cm} X}
\arrayrulecolor{green}\hline
\multicolumn{2}{l}{\textbf{\textcolor{myGreen}{Join group via invite}}}\\
\hline
Tap ``join'' & Join group \\
Tap ``delete'' & Delete invitation \\
\end{tabularx}
\end{center}

\begin{center}
\begin{tabularx}{\textwidth}[t]{p{4cm} X}
\arrayrulecolor{green}\hline
\multicolumn{2}{l}{\textbf{\textcolor{myGreen}{Leave a group}}}\\
\hline
Tap ``Leave Group'' & Display confirmation message (y/n) \\
Accept confirmation message & If leader, display confirmation to choose new leader (y/n), else leave the current group and return to nearby groups \\
Leader accepts confirmation & Display list of group members \\
Leader declines confirmation & Leave current group and return to nearby group list \\
Decline confirmation message & Dismiss message, do not leave group \\
\end{tabularx}
\end{center}

\begin{center}
\begin{tabularx}{\textwidth}[t]{p{4cm} X}
\arrayrulecolor{green}\hline
\multicolumn{2}{l}{\textbf{\textcolor{myGreen}{Group leader disbands group}}}\\
\hline
Leader taps ``disband group'' & Display confirmation message (y/n) \\
Accept confirmation & Display message to all users and return to join group page \\
Decline confirmation & Dismiss message, do not leave group \\
\end{tabularx}
\end{center}

\begin{center}
\begin{tabularx}{\textwidth}[t]{p{4cm} X}
\arrayrulecolor{green}\hline
\multicolumn{2}{l}{\textbf{\textcolor{myGreen}{Enter display name}}}\\
\hline
Empty string & Highlight, no accept \\
Too short & Highlight, no accept \\
All whitespace & Highlight, no accept \\
Valid display name & Accept, trim whitespace, collapse consecutive whitespace \\
\end{tabularx}
\end{center}

\begin{center}
\begin{tabularx}{\textwidth}[t]{p{4cm} X}
\arrayrulecolor{green}\hline
\multicolumn{2}{l}{\textbf{\textcolor{myGreen}{Sign out of app}}}\\
\hline
Tap ``log out'' & Display confirmation message (y/n) \\
Accept confirmation & Log out of the application and display login page \\
Decline confirmation & Dismiss message, do not change application state \\
\end{tabularx}
\end{center}

\begin{center}
\begin{tabularx}{\textwidth}[t]{p{4cm} X}
\arrayrulecolor{green}\hline
\multicolumn{2}{l}{\textbf{\textcolor{myGreen}{Send a group message}}}\\
\hline
Emtpy string & Send button disabled \\
Non-empty string & Message sent to all members and immediately displayed in conversation + saved to storage \\
\end{tabularx}
\end{center}

\begin{center}
\begin{tabularx}{\textwidth}[t]{p{4cm} X}
\arrayrulecolor{green}\hline
\multicolumn{2}{l}{\textbf{\textcolor{myGreen}{See locations of group members}}}\\
\hline
Tap on ``map'' tab & Map is displayed, group member locations displayed on map as pins (zoombed out to display all group members) \\
\end{tabularx}
\end{center}

\begin{center}
\begin{tabularx}{\textwidth}[t]{p{4cm} X}
\arrayrulecolor{green}\hline
\multicolumn{2}{l}{\textbf{\textcolor{myGreen}{Opt-out location updates}}}\\
\hline
Toggle ``enable location updates'' off & Display toast/notify message \\
Toggle ``enable location updates'' on & Display confirmation message (y/n) \\
Accept confirmation & Share user location, begin auto location share \\
Decline confirmation & Toggle ``enable location updates'' off \\
\end{tabularx}
\end{center}

\begin{center}
\begin{tabularx}{\textwidth}[t]{p{4cm} X}
\arrayrulecolor{green}\hline
\multicolumn{2}{l}{\textbf{\textcolor{myGreen}{Receive message}}}\\
\hline
Received message & Unread indicator in tab navigation bar \\
\end{tabularx}
\end{center}

\arrayrulecolor{black}


\section{System Integration Analysis}

System integration testing is a process by which we verify that a newly
introduced piece of code or feature properly integrates with the existing code
and that few errors or bugs arise. This usually involves performing black box
testing on a very specific and well-defined subsystem. In order to accurately
perform these tests such that the subject subsystem is in no way affected by any
other aspect of the project, the unit test will often supply mock data and
mock functionality that behaves according to specifications.

Due to the time constraints of this project, a majority of our testing design
efforts were put towards developing the overal system testing matrix, which
tests the entire system rather than just a specific subset. This allowed us to
devote more time towards development, documentation, and business endevaors. It
is unfortunately a side affect that the subsystems we have built will need to be
revisited and have unit tests applied to them in order for us to verify that
each component is functioning as expected.


\section{Risk Analysis}

When first starting this project, we quickly became aware of a number of risks
that could hinder our progress. Many of these risks were related to our team and
whether or not we would be capable of producing a market-ready product, while
a fair number of these risks were about the technology we were utilizing and
whether or not we could depend on it. The third area of risk was the in the
business as a whole, which is an area in which none of the team was particularly
knowledgable in.


\subsection{Team Risks and Mitigation}

When design and development of this application was first underway, the biggest
risk was that the team would not have the skills or training necessary to
successfully produce the desired product. Each team member was an undergraduate
student studying computer science with little experience in mobile application
design and development. Few of us had any graphic design experience, and none
had worked in a professional capacity on apps.

Additionally, being college students with at most part-time jobs, the
availability of each team member was less than ideal. Each member was required
to divide their attention between multiple concurrent projects and assignments
as well as any jobs or internships they have.

To mitigate the lack of experience and education in mobile app development,
graphic design, and user experience design, each team member spent a large
amount of time researching and studying in detail the behavior and usage of each
technology that they would be working with. Frequently, the team would meet to
share research with each other and educate one another. On multiple occasions,
team members would prepare impromptu presentations to communicate the results
of their effots with the other team members.

Mitigating the time management risk was a more challenging task. To do so, each
team member was forced to cut back on work hours in order to devote more time to
this project in order to keep up with requirements and deadlines. Weekends were often dedicated to working on this project, with many members spending 4-8 hours
a day developing the core product. Frequent meetings with the rest of the team
and strict time schedules helped keep each member on task and the team as a
whole moving forward at a swift and consistent rate.


\subsection{Technology Risks and Mitigation}

The core product of this project depends ehavily on an active internet
connection and a consistent feed of data between third party services in order
to function properly. This means that a major risk that lies beyond our control
is the availability of internet service for our end users as well as service
uptime. Because these services are provided by third parties, our team has
little control over their availability, pricing, and features.

Mitigating this risk is a difficult one, as it involves depending on technology
that we have no guarantee on how it will or will not change. The most we can do
is ensure that our dependencies on these services is not so much that the
project will utterly collapse if one of them becomes unavailable or otherewise
unusable. The best way to do this is to minimize the amount of coupling between
our software and the libraries provided by third parties and increase
separability between the two.

This risk is of particular relevance because it was one of the first risks
identified early on and as a result was addressed before any amount of code was
written. In the end, this proved to be an extremely wise decision, as one of our
third party services, Parse, which is responsible for data storage, retrieval,
account management, and cloud code, will be terminating its services on January,
2017. When this announcement was made, we realized that our initial efforts
towards separability and separability were not in vain and will become useful
shortly. This now introduces a new risk for us to address: finding a backup
database server service.

Fortunately, with the announcement of Parse's closure, the company that operates
the service also announced that the software used to power the service will be
made open source so that we or any other company interested in continuing Parse
service on their applications without needing to rewrite thousands of lines of
code can do so by hosting their own servers. To ensure that we can continue
development of this product without being severely affected by the closure of
Parse, we plan on utilizing another third-party generic server host that will
be capable of handling the capacity of our application using the server software
provided by Parse.


\subsection{Business Risks and Mitigation}

The ultimate goal of this project is to develop a mobile application that can be
released to market. Because of this, one huge aspect of this project was not
just to produce a piece of software that functions well, but also to position
the team and company to prepare for launch and to generate revenue from the
product.

Rather than point out one particular area of the business side of things, we
realized that as a whole we were inexperienced and largely ignorant on what went
into building a successful business. This resulted in virtually every business
decision we made into a potential risk; without fully understanding the
consequences of each decision, it was entirely possible for us to work ourselves
into a corner.

To mitigate this risk, we began reaching out to and taking advice from the
entrepreneurs in residence at our school. These entrepreneurs who have had
past and present business experience were kind enough to share their knowledge
with our team. Decisions such as target marget, how to promote our app, and
alternative sources of revenue are all business aspects that we discussed at
length with these entrepreneurs.



\section{Successes, Issues and Problems}

During the course of our project's development, we have encountered various
successes, issues, and problems. Details on each of these are listed below,
along with a summary on the changes made to our backlog throughout the course of
this project. This section covers successes, issues, and problems for the entire
project, including aspects that are not directly related to testing.


\subsection{Successes in Client Application}

The core deliverable of this project has been the iOS and Android client
applications. These apps were the main focus of our design and development and
are the product that end users will interact with directly. Thus, it was
important to ensure that care and attention was devoted to this aspect of the
project.

During this project, we made great progress in creating functional versions of
the client application for both the iOS and Android mobile phone platforms.
In both versions we successfully incorporated the following features:

\begin{itemize}
	\item Create new account
	\item Log in to existing account using email address
	\item View list of joinable groups
	\item ``Pull-to-refresh'' list of joinable groups
	\item View information on a particualr group
	\item Join an existing group
	\item Create a new group
	\item View own location on a dynamic map
	\item View information on the current group
	\item Leave the current group
	\item Sign out of the application
\end{itemize}

During the later half of this project, the team shifted focus towards developing
exclusively for the Android version of the client, as all team members already
posessed the required tools to work on it. As a result, the Android version of
our client application has an additional set of features compared to that of the
iOS version. The following features exist in the Android version at the time of
this writing:

\begin{itemize}
	\item View dedicated settings screen
	\item Change user display name
	\item View other group members' locations on dynamic map
	\item View list of other members on group info screen
	\item Send instant messages to other group members
	\item Receive instant members from other group members
	\item Group leaders can choose to disband the group
	\item Group leaders can promote other members to replace them as the group leader
	\item Group leaders can remove other members from the group
	\item Users can view pending invitations to join existing groups
	\item Clients automatically fetch group changes periodically
	\item Clients automatically broadcast own location to other group members
\end{itemize}

Because one of the original deliverables of this project was both iOS and
Android client applications, this fragmentation of features is less-than-ideal,
yet ultimately necessary in order to produce at least one functional prototype
before the end of this course. Over the summer, we plan to continue development
of these features. We will also be dividing focus between both versions of the
client so that the iOS version will be caught up to the progress we have made in
the Android version.


\subsection{Successes in Cloud Code}

Our project encompasses both client-side and server-side development. To produce
server functionality, we developed cloud code to run on our Parse data server.
Much of the logic that requires chained queries into the database, cross
references, and multi-row accesses we implemented in cloud code. By moving this
logic to the server, we are able to serve complicated requests while minimizing
network access and increasing effiency and data integrity across threads.


\subsection{Successes in Testing}

The greatest success of our work in unit and system tesing has been in the
development of our testing matrix. Our tesing matrix, while not comprehensive
(as such a matrix requires years of refining to produce), will act as a set of
blueprints when setting up automated tests. At over one fifty test cases
strong, this matrix defines the behavior of our application in many different
possible states. Using this information, we can implement automated unit tests
more efficiently than if we did not have such a matrix.

A secondary success in our testing efforts was in discovering that all of the
necessary tesing frameworks are free and are mature enough to have exhaustive
documentation and support. We have access to a plethora of information that has
aided us in designing and implementing unit and system tests using these native
tools. Additionaly, the tools we are using integrate seamlessly into the IDEs we
are using for development, reducing the number of tools required to use them.

Finally, we have succeeded in designing and implementing a number of UI unit
tests for the iOS version of our client applications. As implementing unit tests
takes a significant time investment, the ones we have built provide a good
foundation upon which to build. For example, many interface tests relate to the
login and signup screens and are used for tesing account creation and login.


\subsection{Issues in Dependencies}

Midway through this project, one of our core dependencies, Parse, which is a
third party service that acts as a database backend and cloud code host,
announced that it would be discontinuing service effective January 31st, 2017.
This means that before the end of the year, our project's backend will need to
be transferred to another service of similar functionality. However, because
Parse will not be discontinuing service until after the end of 2016, we have
prioritized feature development over mitigating this risk, as there is still
time to address this issue.


\subsection{Issues in Testing}

One major issue we encountered when designing unit tests was the fact that none
of the team members have had significant experience doing so. Although a number
of us have had a moderate amount of formal education on the subject through
couse and guest lectures, none have designed or implemented automated tesing in
a project of this magnitude. Thus, an enormous part of testing for this project
was devoted to studying testing methods used in other projects and researching
how to use our tools to build these tests.

The second issue encountered was time constraints. Our goals to turn this
project into a viable product fit for the market required that we focus heavily
on progress towards a working prototype that can be demonstrated by the end of
the semester. Automated tesing, although an important development tool, requires
a large time investment overhead that we could not afford if we desired to meet
requirement deadlines and achieve our critical milestones.

When making the concious decision to prioritize our efforts, we agreed that code
separability and maintainability should not be neglected. These two tennants of
software development are heavilty promoted by automated tesing. By choosing to
prioritize separability and maintainability when designing our software, we are
confident that delaying automated tesing will be less of a hurdle when the time
comes to implement them. Regardless, we are aware that by not investing the time
now, we are accruing ``technical debt'', which means that this task will be more
difficult to accomplish the longer we delay.


\subsection{Problems in Testing}

The unit and system testing section of this project was not without its own set
of problems. Many of these we did not forsee, as they are problems that have
arisen from our tools or our tesing devices, both of which we have limited
control over.

Xcode, the IDE chosen for iOS development underwent an update that apeared to
cause UI tesing on iOS to become undependable. Often unit tests run using this
tool would immediately report failures without apparently running the test
properly. It is possible that this is a configuration error, but it is one that
we have yet to solve if it is. Other Xcode users have reported similar problems
using these features.

Another problem encountered was in using our hardware devices to run automated
tesing. For the most part, the device simulators and emulators are adequate for
running most tests. In some cases, the devices would not cooperate with the
testing frameworks. These devices would either run too slowly for the tests and
sometimes reject the incoming connection to the IDE. In the end, we decided to
forego solving these technical difficulties to focus on product development and
business-related issues.


\subsection{Changes to the Backlog}

